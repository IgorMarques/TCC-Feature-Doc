% Resumo em língua vernácula
\begin{center}
	{\Large{\textbf{Twydi: Uma Ferramenta de Documentação para Equipes de Desenvolvimento de Software}}}
\end{center}

\vspace{1cm}

\begin{flushright}
	Autor: Igor Marques da Silva\\
	Orientador: Prof. Dr. Fernando Marques Figueira Filho
\end{flushright}

\vspace{1cm}

\begin{center}
	\Large{\textsc{\textbf{Resumo}}}
\end{center}

\noindent Compara mídias utilizadas pelos desenvolvedores de software para resolução de problemas. Levanta problemas encontrados com estas mídias. Desenvolve ferramenta capaz de agregar descrições textuais, código, referências externas e \textit{links} com o intuito de melhorar a gestão de conhecimento de equipes de  desenvolvimento de software. Avalia junto a desenvolvedores a ferramenta implementada. Conclui que, apesar de melhoras no quesito de pré-visualização do documento serem necessárias, a ferramenta atende os requisitos de suporte a troca de conhecimento de desenvolvedores através do armazenamento do registro de soluções e compartilhamento de informações e  referências para a equipe.

\noindent\textit{Palavras-chave}: Gestão de Conhecimento, Reuso de Software, Desenvolvimento de Software.
