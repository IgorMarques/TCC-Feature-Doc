% Resumo em língua vernácula
\begin{center}
	{\Large{\textbf{Twydi: Uma Ferramenta de Documentação para Equipes de Desenvolvimento de Software}}}
\end{center}

\vspace{1cm}

\begin{flushright}
	Autor: Igor Marques da Silva\\
	Orientador: Prof. Dr. Fernando Marques Figueira Filho
\end{flushright}

\vspace{1cm}

\begin{center}
	\Large{\textsc{\textbf{Resumo}}}
\end{center}

\noindent Desenvolvedores de software constantemente trocam informações entre si para o aprimoramento contínuo de suas habilidades e resolução de problemas. Muitas dessas informações são perdidas devido ao uso de mídias que não oferecem bom suporte ao armazenamento, recuperação e visualização de informações.
Este trabalho apresenta uma ferramenta que permite aos desenvolvedores de agregar descrições textuais, código, referências externas em artefatos, além de possibilitar a recuperação destes, bem como os estudos realizados ao longo do seu processo de desenvolvimento e avaliação. Apesar de melhorias no quesito de pré-visualização dos artefatos serem necessárias, atende os requisitos de suporte a troca de conhecimento de desenvolvedores através do armazenamento do registro de soluções e compartilhamento de informações e  referências para a equipe. Provê uma base para futuras ferramentas que se proponham a resolver os mesmos problemas.

\noindent\textit{Palavras-chave}: Documentação de Software. Reuso de Software. Desenvolvimento de Software.
