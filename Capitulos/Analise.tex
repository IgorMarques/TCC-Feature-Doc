\chapter{Análise e Discussão Resultados}

Após o fim do período de uso pelos participantes da pesquisa, o autor iniciou a análise das respostas obtidas no questionário aplicado. Todas as perguntas eram opcionais pelo intuito de garantir que apenas respostas de qualidade fossem informadas.

\section{Avaliação da ferramenta}

Seguem as análises sobre cada uma das perguntas do questionário (com exceção da primeira, que era de identificação do participante). Cada participante foi anonimizado e toda referência a um participante sera feita na forma de P\# onde \# identifica o identificador do participante.

\subsection{Sobre o que era seu twydi?}

Os assuntos dos documentos foram variados. Alguns mais simples descreviam como clonar um repositório utilizando \textit{git}\footnote{\url{https://git-scm.com}} (criado por P4). Já P3, por exemplo, redigiu um documento detalhado de acesso à sua API. Outro documento que vale ser mencionado foi o criado por P8, sobre como utilizar uma biblioteca específica na linguagem \textit{Ruby}.

Percebe-se que a ferramenta suporta a criação de documentos com diferentes tipos de complexidade e assunto.

\subsection{No seu dia-a-dia de desenvolvimento, como você resolve os problemas que pelos quais você passa ou como descobre como implementar algo novo pela primeira vez?}

Todos os participantes com exceção de P4 afirmaram utilizar ferramentas de busca como Google, sites de perguntas e respostas como Stack Overflow e contato com outros desenvolvedores como meio de obter auxílio. P4 relatou que apenas utiliza ferramentas de pesquisa \textit{online}.

Nota-se então nas equipes das quais os participantes fazem parte não existem práticas de gestão de conhecimento, tendo em vista de que não houve nenhuma resposta afirmando que havia algum tipo de consulta a algum repositório da equipe.

\subsection{Com relação aos Twydis, qual a utilidade que você vê nos \textit{links} relacionados?}

Com exceção do participante 6 (que não respondeu a questão), todos afirmaram que utilizariam o campo de \textit{links} relacionados como meio de permitir ao leitor aprofundamento sobre o assunto abordado no documento. De acordo com P7, por exemplo: \textit{``Obter mais informações ou detalhes caso seja necessário.''}.

P2 afirmou: \textit{``Embasamento pro que foi escrito, e ainda a possibilidade de aprofundar o assunto.''}. Assim, imagina que a função deste campo seja, além da já mencionada, oferecer maior credibilidade sobre o que está sendo escrito.

Assim, verifica-se que a ferramenta pode também servir de ponto de partida para o aprendizado de outros assuntos além do abordado nos seus documentos presentes, servindo de referencial para outras fontes de conhecimento para a equipe.

\subsection{O que você mudaria na forma como os links relacionados estão implementados atualmente?}

Sobre esta pergunta, P6 e P7 não responderam. Os demais participantes afirmaram que não sentem necessidade de alterações neste campo.

\subsection{E qual a utilidade que você vê nas tags?}

Todos os participantes com exceção de P6 (que não respondeu) afirmaram que a utilidade das \textit{tags} seria de facilitar a busca por outros usuários da aplicação.

\subsection{O que você mudaria na forma como as tags estão implementados atualmente?}

Sobre esta pergunta, P6 e P7 não responderam. P1 sugeriu: ``Faz igual do stackoverflow''. O mecanismo de inserção de \textit{tags} em perguntas do site Stack Overflow exibe sugestões de \textit{tags} baseado no conteúdo escrito em suas perguntas, além de sugerir a grafia das \textit{tags} conforme o usuário as digita (garantindo que são utilizadas sempre \textit{tags} padronizadas pela aplicação). Sem dúvida, uma sugestão interessante que enriqueceria a experiência do usuário da ferramenta.

Os demais participantes afirmaram que não sentem necessidade de alterações neste campo.

\subsection{Que importações de código você utilizou?}

Para a criação do documento, P1 e P4 utilizaram a importação de linhas de arquivo. P5 e P6 utilizaram a importação de Múltiplas Linhas de Arquivo. P8 utilizou a importação de Arquivo Inteiro. Os demais participantes não responderam à questão. Destes, P2 afirmou como resposta da pergunta seguinte não ter utilizado o mecanismo de importação. Importações de \textit{commit} e \textit{pull-request} não foram utilizadas.

\subsection{Que importação de código você achou mais útil? Por quê?}

Todos os participantes informaram respostas bastante variadas. P1 afirmou não ver utilidade na ferramenta de importação. P2 afirmou: \textit{``Não utilizei nenhuma, mas talvez a mais útil seria as relacionadas com linhas de arquivo, principalmente se puderem ser gists.''}. Gists\footnote{\url{https://gist.github.com}} funcionam como repositórios de um arquivo só hospedados no GitHub. São comumente utilizados por desenvolvedores para armazenar \textit{scripts} de código, arquivos de configuração ou notas pessoais. O suporte da aplicação desenvolvida para esse tipo de armazenamento de código pode ser uma funcionalidade útil para outros desenvolvedores. P5 afirmou que importação de um \textit{commit} é a mais útil em sua opinião. P8 respondeu: \textit{``Considero o modo de múltiplas linhas porque permite uma maior variedade de combinações para a resposta''}.
% IM: perguntei dps e ele quis dizer que é pq varias linhas exemplificam melhor que uma só. Posso complementar isso?

P4 afirmou não ter conseguido utilizar a importação. P3 e P7 não responderam à pergunta.

\subsection{Algum resultado da importação de código ficou ruim/ confuso? Por quê?}

P4 e P6 afirmaram terem encontrado problemas com a importação. P4 informou: \textit{``Não consegui utilizar importação, pois não entendi bem como funciona.''}, apesar das instruções presentes na tela de criação de documento. P6 afirmou teve problemas para importar um arquivo \textit{``Leia-me''} em formato \textit{markdown}, havendo a má detectação de sua linguagem. De fato a ferramenta não foi pensada com este tipo de arquivo em mente, o que jusifica os problemas encontrados. Futuras versões da mesma podem tratar melhor deste caso.

Os demais participantes não responderam ou afirmaram não terem encontrado problemas com a importação.

\subsection{No caso da importação de \textit{commits} e \textit{pull requests}, é mais importante: }

Metade dos participantes afirmaram que é mais importante visualizar o código adicionado e removido pela ação. Esta forma é a já feita pela aplicação atualmente e, graças a estas respostas, deve se manter assim. Os demais participantes não responderam ou afirmaram não saber o que é mais importante.

\subsection{Como você utilizaria a ferramenta no seu dia-a-dia?}

Esta pergunta também obteve respostas variadas. P1, P2 e P8 afirmaram que utilizariam como forma de registrar como estes resolveram algum problema já enfrentado. P2 afirmou: \textit{``Relembrar alguma implementação que eu fiz, ou alguma boa prática que achei interessante.''}. P3, P6 e P8 afirmaram que utilizariam para a divulgação de tutoriais ou soluções para suas equipes. P7 afirmou que utilizaria a aplicação para gerar arquivos de ``Leia-me'', enquanto P5 afirmou que não utilizaria com frequência. P4 afirmou que utilizaria, segundo o próprio \textit{``Em projetos de disciplinas.''}.

Assim, os casos de uso mais mencionados pelos participantes estão relacionados intimamente com políticas de gestão do conhecimento, área na qual se percebe deficiência de acordo com o notado através da pergunta ``No seu dia-a-dia de desenvolvimento, como você resolve os problemas que pelos quais você passa ou como descobre como implementar algo novo pela primeira vez?''

\subsection{O que pode ser melhorado na ferramenta para facilitar a escrita de documentos?}

P3, P5, P6 e P7 afirmaram que funcionalidades relacionadas ao auxílio na formatação do documento como ferramentas visuais para formatação de texto ou pré-visualização do documento gerado. P1 e P8 afirmaram que não acham que haja nada a ser melhorado. P2 fez uma sugestão semelhante a de P1 na pergunta ``O que você mudaria na forma como as tags estão implementados atualmente?'': \textit{``Sugestões de tags baseados no conteúdo do twydi.''}.
% FF: usar comando \label para referenciar outras seções

Dessa forma, para futuras versões da aplicação, essas melhorias no formulário de implementação do documento devem ser implementadas.

\subsection{Além dos campos já presentes em um Twydi, que outras informações você acrescentaria?}

A maioria dos participantes afirmou não ter nenhum campo a mais em mente. P8 afirmou: \textit{``Talvez a possibilidade de adicionar imagens''}.

\subsection{Houve alguma dificuldade para o preenchimento deste questionário ou para o entendimento das atividades solicitadas?}

Todos os participantes afirmaram não terem encontrado dificuldades para o preenchimento do questionário.

\section{Conclusões Obtidas e Respostas das Perguntas de Pesquisa}

Com base nas respostas obtidas, foi possível obter respostas para as perguntas de pesquisa inicialmente propostas.

\subsection{1. A ferramenta proposta oferece suporte à maneira como conhecimento é trocado em equipes de software?}

A ferramenta, de acordo com os participantes não substitui totalmente a forma como conhecimento é trocado em suas respectivas equipes, porém, atua como um complemento a esta forma. Foi afirmado que a ferramenta pode atuar como agregador e organizador de informações já conhecidas pela equipe. Desta forma, é possível afirmar que \textbf{sim, a ferramenta oferece suporte à troca de conhecimento entre desenvolvedores}.

\subsection{2. Que outras funcionalidades podem ser agregadas a ferramenta para melhorar este suporte, caso exista?}

De acordo com as respostas obtidas e tendo em vista de que o suporte existe, as funcionalidades mais requisitadas são de \textbf{melhor suporte a pré-visualização e formatação e dos documentos}, mais especificamente como botões de adição de formatação ao texto inserido. Melhorias com relação ao mecanismo de \textit{tags} também foram solicitados, porém, em escala reduzida.

\subsection{3. Quais são os casos potenciais de uso da ferramenta segundo os desenvolvedores?}

Dois casos de uso foram mencionados por múltiplos participantes, sendo o primeiro \textbf{divulgação de tutoriais ou soluções para suas equipes} e o segundo \textbf{registro pessoal sobre como o próprio desenvolvedor solucionou determinado problema}. Assim, a ferramenta possui utilidade tanto relacionada a difusão de informação para a equipe quanto relacionada ao registro e organização de conhecimento pessoal.

\section{Limitações do Estudo e Lições Aprendidas}

Este estudo possui possíveis limitações. Uma delas sendo o curto período de tempo no qual a ferramenta esteve disponível (três semanas). Com mais tempo, certamente haveria a oportunidade de se estabelecer uma cultura de uso da aplicação dentro dos contextos analisados e quesitos relacionados a recuperação de informação poderiam ser avaliados.

Além disso, houve uma participação limitada de voluntários e de apenas dois escopos distintos (alunos e membros da empresa júnior). Um escopo de participantes de empresas de desenvolvimento de software já estabelecidas certamente poderia enriquecer ainda mais os resultados da pesquisa e agregar mais melhorias para futuras versões da ferramenta.

% IM: Não falei sobre cultura de adaptação do uso da ferramenta
