\chapter{Considerações Finais}

Ao longo do desenvolvimento desse estudo, foi proposta a elaboração de uma ferramenta que oferecesse suporte a gestão de informação dentro do contexto de equipes de desenvolvimento de software.

Rotatividade de pessoal e falta de registro de conhecimento obtido, aliado a alta necessidade de reuso na área da engenharia de software e a precariadade das mídias atuais deixam clara a demanda por uma ferramenta como a proposta.

Estudo de ferramentas semelhantes e inquéritos contextuais realizados por desenvolvedores ajudaram na elicitação e validação de requisitos para a ferramenta.

A ferramenta atua como um catálogo de implementações, descrições textuais passo-a-passo e referências externas na forma de \textit{links}, permitindo a desenvolvedores criarem e recuperarem artefatos produzidos pela sua equipe.

Após sua implementação, um questionário de avaliação foi aplicado com desenvolvedores e foi constatado que a ferramenta elaborada oferece suporte à troca de conhecimento entre desenvolvedores. Porém um melhor suporte a pré-visualização e formatação e dos documentos podem aprimorar sua eficiência no dia-a-dia. Sobre sua utilidade foi constatado que a divulgação de tutoriais ou soluções para suas equipes e registro pessoal sobre como o próprio desenvolvedor solucionou determinado problema seriam casos de uso nos quais utilizariam a ferramenta em sua rotina.

Futuros estudos podem validar a aplicabilidade da ferramenta em outros contextos e por períodos maiores de tempo, com foco maior na avaliação da eficiência de recuperação de informações armazenadas na ferramenta quando comparada a outras mídias utilizadas pelos desenvolvedores de software da atualidade.
