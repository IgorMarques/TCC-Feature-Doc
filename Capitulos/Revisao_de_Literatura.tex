\chapter{Revisão de Literatura}

Este capítulo trata das referências utilizadas como base para o trabalho nas áreas de: uso de mídias pelo engenheiro de software, rotatividade de pessoal, gestão e recuperação de informações e reuso.

\section{Uso das mídias pelo engenheiro de software}

A atividade de desenvolvimento de software requer que seus profissionais se mantenham constatemente atualizados, devido ao fluxo constante de inovações que surgem diariamente~\cite{Singer2014}.

Uma das formas que desenvolvedores utilizam para se manterem atualizados são as redes sociais~\cite{Treude2012}~\cite{Storey2014} além de seus próprios colegas de trabalho, tendo em vista que desenvolvedores frequentemente recorrem a própria equipe para obter ajuda~\cite{Weinberg1998}.

Sobre as mídias em geral, Clark e Brennan~\cite{Clark1991} estabeleceram que estas podem apresentar as seguintes características.

\begin{itemize}
  \item Copresença: está no mesmo ambiente físico.
  \item Visibilidade: está visível aos participantes da comunicação.
  \item Audibilidade: é possível de ser ouvida pelos participantes da comunicação.
  \item Contemporaneidade: a mensagem é recebida imediatamente.
  \item Simultaneidade: Ambos os participantes podem enviar e receber mensagens.
  \item Sequencialidade: turnos não podem sair de sequência.
  \item Revisibilidade: permite rever as mensagens enviadas.
  \item Revisabilidade: permite rever uma mensagem antes de ser enviada.
\end{itemize}

A comunicação com colegas de trabalho face-a-face comparado a outras mídias, provê uma série de vantagens, conforme apresentado na Tabela 1.

\begin{table}[h!]
\centering
\caption{Características das mídias de comunicação (adaptado de Olson et. al, 2000)}
\label{my-label}
\begin{tabular}{l|c|c|c|c|c|c|c|c}
\hline
Mídia                & \begin{turn}{90}Copresença\end{turn} &\begin{turn}{90} Visibilidade\end{turn} & \begin{turn}{90}Audibilidade\end{turn} & \begin{turn}{90}Contemporaneidade \, \end{turn} & \begin{turn}{90}Simultaneidade\end{turn} & \begin{turn}{90}Sequencialidade\end{turn} &\begin{turn}{90}Revisibilidade\end{turn} & \begin{turn}{90}Revisabilidade\end{turn} \\ \hline
Face-a-face          &   x     &      x        &    x          &          x        &        x        &         x    &                &                \\ \hline
Telefone             &           &       &     x         &           x        &         x       &         x        &                &                \\ \hline
Vídeoconferência     &        &       x   &          x    &       x            &         x       &       x          &                &                \\ \hline
Chat de duas vias    &                                 &                                   &              &           x        &          x      &        x         &        x        &          x      \\ \hline
Secretária Elerônica &                                 &                                   &     x         &                   &                &                 &        x        &                \\ \hline
E-mail               &                                 &                                   &              &                   &                &                 &      x          &     x           \\ \hline
Carta                &                                 &                                   &              &                   &                &                 &       x         &       x         \\ \hline
\end{tabular}
\caption*{fonte: Olson et. al, 2000 (adaptado pelo autor)}
\end{table}

Apesar destas vantagens, a comunicação face-a-face não apresenta uma importante característica para a gestão do conhecimento de uma equipe: a revisibilidade. Tanto ela quanto a revisabilidade dão auxílio a ambos comunicadores permitindo que formulem cuidadosamente a mensagem que desejam transmitir, além de permitir várias chances de revisar o que já foi trocado ~\cite{Olson2000}. A revisibilidade, em especial, possui relação direta com a gestão e recuperação de informação. Informações já conhecidas pela equipe e já trocadas tem enorme valor, principalmente em equipes com alta rotatividade de pessoal.

Ferramentas de \textit{chat}, por sua vez, apresentam a caracterísica da revisabilidade, mas ainda sim possuem problemas de recuperação fácil de informações trocadas nela (variando de ferramenta para ferramenta). Informações cruciais são trocadas nessa mídia, mas muitas vezes são impossíveis de serem recuperadas devido ao alto volume de mensagens trocadas.

Dessa forma, mídias com um bom suporte a recuperação são de extrema importância para equipes de desenvolvimento de software.

\section{Rotatividade de pessoal}

Rotatividade de pessoal é inerente a qualquer área de atuação da indústria.
Sua ocorrência acarreta em custos de transferência de conhecimento e treinamento~\cite{Hall2008}, possuindo relação com sucesso ou fracasso de projetos, incluindo aqueles que envolvem o desenvolvimento de software~\cite{Hall2008}.

Em equipes de desenvolvedores de software, a partida de um membro pode acarretar em perda de conhecimento muitas vezes precioso para a equipe em casos onde não haja nenhuma outra forma de registro em mídia de tais informações.

Em geral, novos membros da equipe passam por um período de adaptação onde devem aprender padrões, práticas e ferramentas utilizadas pelo grupo. Além disso, os novatos estão sucetíveis a outras barreiras como quebras de expectativas, problemas com recepção, má configuração de ambiente de trabalho e curva de aprendizado~\cite{Steinmacher2015}.

Sobre a curva de aprendizado, esta tem peso ainda maior em casos específicos como desenvolvedores iniciantes (em geral chamados de ``desenvolvedores júnior'', no mercado), pois existe a necessidade em se adquirir conhecimento sobre as tecnologias utilizadas pela equipe ao qual foi recém integrado.

Todo esse conhecimento é em geral transmitido oralmente, por via escrita ou repasse de referências (documentação, links externos, etc)~\cite{Storey2014, Olson2000, CubraniC2004}. Conforme mencionado anteriormente algumas mídias não possuem persistência ou possuem métodos precários de recuperação.

Sendo assim, uma ferramenta que estimule seus usuários a registrarem informações fundamentais para a equipe tende a evitar tanto que informações sejam perdidas no caso de partidas quanto facilitar o aprendizado e adaptação de novos membros à equipe.

\section{Gestão e recuperação de informação}

A gestão de conhecimento tem por objetivos a aquisição de novos conhecimentos, como lidar com o conhecimento existente para garantir seu uso futuro, passando pelo seu armazenamento e difusão, bem como provendo estratégias aplicáveis para novos contextos~\cite{Bjornson2008}.

Uma boa gestão de informação e facilidade de se recuperar informações permitem que organizações de desenvolvimento de software se mantenham inovadoras e competitivas~\cite{Rabelo2015}.

É possível ver então que uma equipe de desenvolvimento que apresente alguma forma de gestão de informações pode garantir, facilitar e estimular, por exemplo, o reuso de soluções já conhecidas de problemas recorrentes.

\section{Reuso em engenharia de software}

Possibilidade de reuso de software é uma das caracterísitcas chaves de um código de qualidade. Reuso de software pode ser definido como o processo de construir sistemas de software a partir de software já existente em vez de ``do zero''~\cite{Krueger1992}.

Vale lembrar que os tipos de artefatos que podem ser reusados não estão limitados somente a código-fonte, mas também podem incluir \textit{design} de estruturas, implementação de módulos, especificações, documentação e etc~\cite{Freeman1993}.

Diversos cientistas da computação ainda vêem o reuso como uma potencial meio para melhorar as práticas da engenharia de software, além de suas vantagens serem amplamente conhecidas~\cite{Krueger1992}.

É notável então que uma boa forma de garantir a gerência de informações visando o reuso de conteúdo produzido por uma equipe pode trazer melhorias para a qualidade dos projetos produzidos pela mesma.

\section{GitHub}

GitHub\footnote{\url{http://github.com}} é um serviço web de repositórios de código utilizando primariamente Git como controle de versão~\cite{Figueira2015}. Atualmente possui cerca de onze milhões e meio de desenvolvedores e vinte e oito milhões de repositórios cadastrados\footnote{\url{https://github.com/about/press}}.

O que destaca o GitHub das demais ferramentas de repositório online é o grande apelo deste para a disponibilidade de projetos de código aberto.

É comum que empresas de desenvovimento de software criem organizações dentro do GitHub. Organizações funcionam como um agregado de usuários com acesso comum aos mesmos repositórios. Cada usuário continua mantendo uma conta própria para si.

Por ser o maior centro de hospedagem de código da atualidade~\cite{Gousios2012}, se mostra uma excelente plataforma para se obter exemplos de código de diversas linguagens e finalidades a serem compartilhados entre desenvolvedores.
