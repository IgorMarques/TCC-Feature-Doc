\chapter{Revisão de Literatura}

Este capítulo aborda a fundamentação teórica deste trabalho.

\section{Uso das mídias pelo engenheiro de software}

A atividade de desenvolvimento de software requer que seus profissionais se mantenham constatemente atualizados, devido ao fluxo constante de inovações que surgem diariamente. [CARECE DE FONTES]

Uma das formas que desenvolvedores utilizam para se manterem atualizados são as redes sociais [CARECE DE FONTES] além de seus próprios colegas de trabalho [CARECE DE FONTES].

[FALAR DAS LIMITAÇÕES DAS REDES SOCIAIS DE SOBRE COMO COMUNICAÇÃO FACE A FACE É MELHOR].

[FALAR DOS CUSTOS DA COMUNICACAO FACE A FACE E SE SUAS LIMITACOES]

Nota-se então a importância do fator pessoal para o aprendizado dos desenvolvedores de software modernos.

% \section{Mentoria de desenvolvedores}

% IM: Falar sobre isso?

\section{Rotatividade de pessoal}

Rotatividade de pessoal é inerente a qualquer área de atuação da indústria [CARECE DE FONTES]. Sua ocorrência acarreta em [PROBLEMAS].

Em equipes de desenvolvedores de software, a partida de um membro pode acarretar em [FALAR DE COISAS COMO PERDA DE INFORMAÇÕES] caso não haja nenhuma forma de registro em mídia da mesma.

Em geral, novos membros da equipe passam por um período de adaptação onde devem aprender padrões, práticas e ferramentas utilizadas pelo grupo [CARECE DE FONTES]. Em casos específicos, como desenvolvedores iniciantes (em geral chamados de \"desenvolvedores júnior\", no mercado), ainda existe a necessidade em se adquirir conhecimento sobre as tecnologias utlizadas pela equipe ao qual foi recém integrado.

Todo esse conhecimento é em geral transmitido via oral [CARECE DE FONTES] ou por mídias que não oferecem um bom suporte à sua manutenção [CARECE DE FONTES].

Sendo assim, uma ferramenta que estimule seus usuários a registrarem informações fundamentais para a equipe tende a evitar tanto que informações sejam perdidas no caso de partidas quanto facilitar o aprendizado e adaptação de novos membros à equipe.

\section{Gestão e recuperação de informação}

A gestão de informações é a área [ARRUMAR DEFINIÇÃO].

Uma boa gestão de informação e facilidade de se recuperar informações permitem que [ARRUMAR BENEFÍCIOS].

É possível ver então que uma equipe de desenvolvimento que apresente alguma forma de gestão de informações pode garantir facilitar e estimular, por exemplo, o reuso de soluções já conhecidas de problemas recorrentes.

\section{Reuso em engenharia de software}

Reuso é uma das caracterísitcas chaves de um código de qualidade. Sua definição é [ARRUMAR DEFINIÇÃO DE REUSO].

Bons engenheiros de software procuram desenvolver código com boa possibilidade de reuso, caso seja conveniente [CARECE DE FONTES].

É notável então que uma boa forma de garantir a gerência de informações visando o reuso de conteúdo produzido por uma equipe pode trazer melhorias para a qualidade do código produzido pela mesma.
