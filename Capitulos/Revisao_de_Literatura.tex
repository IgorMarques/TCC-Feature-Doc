\chapter{Revisão de Literatura}

\section{Uso das mídias pelo engenheiro de software}

A atividade de desenvolvimento de software requer que seus profissionais se mantenham constatemente atualizados, devido ao fluxo constante de inovações que surgem diariamente ~\cite{Singer2014}.

Uma das formas que desenvolvedores utilizam tal são as redes sociais ~\cite{Treude2012}~\cite{Storey2014} além de seus próprios colegas de trabalho, tendo em vista que desenvolvedores frequentemente recorrem a própria equipe para obter ajuda~\cite{Weinberg1998}.

% Fred Brooks: The Mythical Man-Month
% IM: não consegui achar algo sobre aprendizado dentro da própria equipe nele.

Sobre as mídias em geral, Clark e Brennan (1991) estabeleceram que estas podem apresentar as seguintes características:

\begin{itemize}
  \item Copresença: está no mesmo ambiente físico
  \item Visibilidade: está visível aos participantes da comunicação
  \item Contemporaneidade: a mensagem é recebida imediatamente
  \item Simultaneidade: Ambos os participantes podem enviar e receber mensagens
  \item Sequencialidade: turnos não podem sair de sequencia
  \item Revisibilidade: permite rever as mensagens enviadas
  \item Revisabilidade: permite rever uma mensagem antes de ser enviada
\end{itemize}

A comunicação com colegas de trabalho face-a-face comparado a outras mídias, provê uma série de vantagens, conforme apresentado por Olson et al (2000)(?) no quadro abaixo.

https://www.dropbox.com/s/3aq85imoew7vidv/Screenshot%202015-10-27%2013.59.20.png?dl=0

Porém, não apresenta uma importante características para a gestão do conhecimento de uma equipe: a revisibilidade.

Ferramentas de \textit{chat}, por sua vez, apresentam esta caracterísica, mas ainda sim possuem problemas de recuperação fácil de informações trocadas nela (variando de ferramenta para ferramenta).

\section{Rotatividade de pessoal}

Rotatividade de pessoal é inerente a qualquer área de atuação da indústria [CARECE DE FONTES].
Sua ocorrência acarreta em [PROBLEMAS].
% staff turnover (software development), Employee turnover, turnover open source development

Em equipes de desenvolvedores de software, a partida de um membro pode acarretar em [FALAR DE COISAS COMO PERDA DE INFORMAÇÕES] caso não haja nenhuma forma de registro em mídia da mesma.
% knowledge management, tacit knowledge, software development (organization or team)

Em geral, novos membros da equipe passam por um período de adaptação onde devem aprender padrões, práticas e ferramentas utilizadas pelo grupo [CARECE DE FONTES].
% 5 - PEDIR A FERNANDO - Rabelo2015
% Knowledge management and organizational culture in a software organization: a case study
% https://scholar.google.com/citations?view_op=view_citation&hl=en&user=I8o8rfoAAAAJ&sortby=pubdate&citation_for_view=I8o8rfoAAAAJ:maZDTaKrznsC


% 6 - Social Barriers Faced by Newcomers - Steinmacher2015

Em casos específicos, como desenvolvedores iniciantes (em geral chamados de \"desenvolvedores júnior\", no mercado), ainda existe a necessidade em se adquirir conhecimento sobre as tecnologias utlizadas pela equipe ao qual foi recém integrado.

Todo esse conhecimento é em geral transmitido oralmente [CARECE DE FONTES] ou por mídias que não oferecem um bom suporte à sua manutenção [CARECE DE FONTES].
% neste parágrafo você explicita quais são os problemas das abordagens atuais. ao mesmo tempo, você justifica o seu trabalho.

Sendo assim, uma ferramenta que estimule seus usuários a registrarem informações fundamentais para a equipe tende a evitar tanto que informações sejam perdidas no caso de partidas quanto facilitar o aprendizado e adaptação de novos membros à equipe.

\section{Gestão e recuperação de informação}

A gestão de informações é a área [ARRUMAR DEFINIÇÃO].
% 5 - PEDIR A FERNANDO - Rabelo2015
% Knowledge management and organizational culture in a software organization: a case study


Uma boa gestão de informação e facilidade de se recuperar informações permitem que [ARRUMAR BENEFÍCIOS].

É possível ver então que uma equipe de desenvolvimento que apresente alguma forma de gestão de informações pode garantir facilitar e estimular, por exemplo, o reuso de soluções já conhecidas de problemas recorrentes.

% \section{GitHub}

% 7 - A study on the geographical distribution - Figueira2015

\section{Reuso em engenharia de software}

Reuso é uma das caracterísitcas chaves de um código de qualidade. Sua definição é [ARRUMAR DEFINIÇÃO DE REUSO].
% ver o livro da Gang of Four

Bons engenheiros de software procuram desenvolver código com boa possibilidade de reuso, caso seja conveniente [CARECE DE FONTES].

É notável então que uma boa forma de garantir a gerência de informações visando o reuso de conteúdo produzido por uma equipe pode trazer melhorias para a qualidade do código produzido pela mesma.
