\chapter{Métodos}

% REVISAR
Este trabalho prevê o desenvolvimento da ferramenta de documentação de implementação funcionalidades. Nela, será possível vincular recursos do repositório do projeto no GitHub\footnote{\url{www.github.com}} (\textit{issues}, \textit{commits}, etc) a requisitos e suas implementações, bem como outros referenciais (\textit{links} para perguntas no Stack Overflow\footnote{\url{www.stackoverflow.com}}, desenhos, por exemplo) de forma a gerar um guia ou tutorial de como realizar tal implementação novamente no futuro.

% REVISAR (ADICIONAR ALUNOS DE FERNANDO)
A empresa 4Soft terá participação significativa em todo o estudo, desde a concepção até no uso da ferramenta.

As etapas do estudo foram:

\section{Estudo de Aplicações Existentes}

% REVISAR

Será feita uma busca por aplicações que realizem atividades semelhantes às propostas. Suas limitações serão analisadas pela equipe de pesquisa e um panorama inicial será traçado de modo que a ferramenta proposta possa suprir as necessidades iniciais e as limitações encontradas.

\section{Inquérito contextual}

% REVISAR (ADICIONAR ALUNOS DE FERNANDO)
Entrevistas e seções de \textit{brainstorm} serão feitas com os participantes da empresa júnior mencionada. Será analisado como se dá seu processo de trabalho e como pode se dar o fluxo de atividades na ferramenta. As informações necessárias serão coletadas através de entrevistas e aplicação de questionários de satisfação.


% IM: vale a pena deixar?
% \section{Sessões de interpretação da equipe}
%
% 	Reuniões com a equipe de pesquisadores que trabalharão no projeto serão feitas para definir com mais detalhes o escopo da ferramenta, bem como será sua arquitetura e implementação.

\section{Prototipação e implementação da ferramenta}

% REVISAR
Nesta etapa, inicialmente, protótipos de baixa fidelidade serão elaborados pelo pesquisador. Posteriormente, serão expostos a todos os participantes do projeto (empresa júnior e pesquisadores) e seu \textit{feedback} será colhido e analisado. A partir daí, a aplicação passará para a etapa de implementação seguindo processo iterativo e incremental de desenvolvimento de software.

\section{Implantação e observação do uso da ferramenta}

% REVISAR (ADICIONAR ALUNOS DE FERNANDO)
A ferramenta então estará disponível para uso de todos os membros da empresa 4Soft. A adesão dos desenvolvedores a ferramenta será analisada nesta etapa, bem como seu uso monitorado (quantidade de artefatos de documentação criados, por exemplo).

\section{Avaliação da ferramenta}

% REVISAR (ADICIONAR ALUNOS DE FERNANDO)
Ao fim do período anterior, uma nova bateria de entrevistas e aplicação de questionários de satisfação serão realizadas para avaliar qualitativamente como se deu a utilização da ferramenta, bem como se deram os efeitos de seu uso e a satisfação de seus usuários.
A versão final do trabalho com os resultados será então redigida.
