\chapter{Projeto de Solução}

Com base nos requisitos então elicitados, pôde-se ter uma noção clara de como seria feita a implementação da aplicação e este processo teve início. Ao longo dele, \textit{feedback} a respeito da estética da ferramenta foi constantemente colhido com a equipe de desenvolvedores da empresa júnior bem como de parceiros do grupo de pesquisa do qual o autor faz parte.

O intuito da ferramenta é permitir a criação de documento vinculem recursos do repositório do projeto no GitHub (arquivos, \textit{commits} e \textit{pull requests}), bem como outros referenciais (\textit{links} para perguntas no Stack Overflow, tutoriais ou \textit{blogs}, por exemplo) de forma a gerar um guia ou tutorial de como realizar tal implementação novamente no futuro.

\section{Listagem de Documentos}

Na tela inicial da aplicação é possível ver uma lista com todos os documentos já criados. Cada documento é exibido com seu título, descrição, lista de \textit{tags} e autor (caso tenha se identificado durante a criação do documento).

https://www.dropbox.com/s/4rput1kdwxxiknj/Screenshot\%202015-11-05\%2016.08.17.png?dl=0

Ao fim da lista de documentos, é possível ver quantos documentos se encontram nela e também uma lista das \textit{tags} presentes, incluindo sua quantidade de ocorrências de cada dentro do grupo de documentos exibidos.

https://www.dropbox.com/s/bdhexdhgwrgo6fl/Screenshot\%202015-11-05\%2016.08.28.png?dl=0

ou

https://www.dropbox.com/s/us93gna8m8k4hkb/Screenshot\%202015-11-05\%2016.08.48.png?dl=0

ou

https://www.dropbox.com/s/3wm1tq2s6gqsrnq/Screenshot\%202015-11-05\%2016.08.59.png?dl=0

\section{Criação de um Documento}

Qualquer usuário da aplicação pode criar um documento, mesmo sem estar autenticado na mesma.
% IM: tá ruim esse parágrafo, eu sei.

https://www.dropbox.com/s/nii8vbjzphgce0r/Screenshot\%202015-11-11\%2013.28.10.png?dl=0

\subsection{Elementos de um documento}

Um documento é composto por 5 elementos, todos informados pelo seu autor:

\begin{itemize}
  \item Título: Título do documento. Deve fornecer uma noção sobre o que aquele documento descreve.
  \item Descrição: Um resumo sobre o que o documento descreve. Descritivo o suficiente para que outros usuários não tenham que ler o documento inteiro para compreender sobre o que ele trata.
  \item Implementação: Procedimento sobre como implementar ou executar a funcionalidade descrita. Possui suporte a linguagem de marcação e importação de código armazenado em repositórios públicos e privados do GitHub.
  \item Links Relacionados: Lista de links relevantes para o documento, cada um com descrição própria. Podem conter referências utilizadas para elaborar o documento ou material mais avançado sobre o assunto.
  \item \textit{Tags}: Lista de \textit{tags} sobre o assunto do documento. Utilizada para facilitar sua recuperação posteriormente tanto pelo autor quanto por outros usuários.
\end{itemize}

Sobre o elemento Implementação, a próxima subseção o descreve em detalhes.

\subsection{Implementação}

Elemento mais importante de um documento e é utilizado para que o usuário descreva um passo-a-passo sobre
como realizar a implementação de uma funciondalidade, ou deixe registrado determinado procedimento. Dentre suas funcionalidades que permitem um uso aprimorado por parte de equipes de desenvolvimento de software está a importação de código hospedado no GitHub.

% IM: VERIFICAR SE ISSO ESTÁ OK \/
No GitHub é possível acessar cada arquivo de um determinado projeto através de URLs simples e que seguem a própria estrutura de diretórios do projeto em si.

Por exemplo, um arquivo com caminho ``app/models/example.rb'' se encontra URL ``github.com/usuario/projeto/blob/master/app/models/doc.rb''.

O arquivo é exibido utilizando uma formatação própria da linguagem, facilitando sua visualização:

% IM: descobrir como botar imagem
https://www.dropbox.com/s/qgrnw4s0impnps2/Screenshot\%202015-10-27\%2016.03.45.png?dl=0

A partir dessa estrutura de URLs, foi elaborado um mecanismo para importar código no editor de texto da implementação toda vez que um \textit{link} do GitHub é colado. Uma descrição sobre os tipos de importações possíveis e seus respectivos resultados encontra-se na Tabela 2.

\begin{table}[]
\centering
    \caption{Tipos de importação suportadas pela aplicação}
    \label{my-label}
    \begin{tabular}{p{3cm} | p{7cm} | p{5cm}}
    \hline
    Tipo da Importação & Exemplo de link & Resultado \\ \hline
    Arquivo & https://github.com/coopera/that-s-the-way-you-do-it/blob/master/app/models/doc.rb & Conteúdo Inteiro do Arquivo é importado. \\ \hline
    Linha de Arquivo & https://github.com/coopera/that-s-the-way-you-do-it/blob/master/app/models/doc.rb\#L1 & Apenas a linha informada é importada. \\ \hline
    Intervalo de Linhas de Arquivo & https://github.com/coopera/that-s-the-way-you-do-it/blob/master/app/models/doc.rb\#L5-L7 & O conteúdo intervalo entre as inclusivo entre linhas é importado. \\ \hline
    \textit{Commit} & https://github.com/coopera/that-s-the-way-you-do-it/commit/d7f365db9777b4cb6e1c5799 a2e431c58aaf3a19 & Um \textit{diff} contendo o conteúdo adicionado e removido no \textit{commit} é importado. \\ \hline
    \textit{Pull Request} & https://github.com/coopera/that-s-the-way-you-do-it/pull/7 & Um \textit{diff} contendo o conteúdo adicionado e removido em todos os \textit{commits} do \textit{pull request} é importado.  \\ \hline
\end{tabular}
\caption*{fonte: o autor}
\end{table}

% IM: Adicionar imagens com exemplos de cada um

Todo código importado para aplicação já vem formatado como código na linguagem de marcação utilizada. Caso o arquivo importado seja de uma extensão conhecida, é adicionada a marcação para código da linguagem referente àquela extensão. Isso diminui o trabalho do autor para a formatação do texto, permitindo que foque em outros aspectos do documento, como seu texto.

O código importado pode ser livremente modificado tendo em vista de que é texto-puro, permitindo que o autor faça modificações caso ache conveniente.

Além disso, o \textit{link} original colado é mantido para que futuros usuários que consultem o documento possam acessar o código-fonte original disponível no seu respectivo repositório no GitHub.

Sobre a linguagem de marcação utilizada, a próxima subseção a descreve, bem como suas vantagens para o contexto.

\subsection{Suporte a \textit{Markdown}}

Uma linguagem de marcação permite que os autores dos documentos possam formatar seus textos com cabeçalhos, listas, tabelas, entre outros elementos.

Markdown\footnote{\url{https://daringfireball.net/projects/markdown/}} é uma linguagem de marcação amplamente utilizada na web atualmente. Possui sintaxe simples e é utilizada em campos textuais de sites e serviços amplamente utilizados por desenvolvedores de software como os já citados GitHub e Stack Overflow.

Assim, foi optado por adicionar suporte a tal linguagem devido a familiaridade dos desenvolvedores modernos com a mesma.

Dessa forma é permitido aos autores maior clareza e poder de expressão com relação aos documentos criados, além de garantir uma padronização em sua formatação (como de tipografia, por exemplo).

\subsection{Exibição do Documento}

Uma vez que o documento é criado, sua visualização fica disponível para todos os usuários. As imagens a seguir mostram como um documento fica exibido dentro da aplicação.

https://www.dropbox.com/s/dh6fao0hidf6dxw/Screenshot\%202015-11-11\%2013.33.45.png?dl=0

https://www.dropbox.com/s/m4x930cld84ga13/Screenshot\%202015-11-11\%2013.34.03.png?dl=0

Enfatiza-se que o elemento de implementação é propriamente renderizado de acordo com a marcação utilizada pelo autor.

\section{Recuperação de Documentos}

Todos os documentos da aplicação são rastreáveis por qualquer desenvolvedor. Existem duas formas de fazer esta busca.

\subsection{Através de \textit{Tags}}

Ao se clicar em uma tag de um documento ou em uma tag presente na lista de tags que se encontra ao fim da lista de documentos, é possível se recuperar todos os documentos que apresentam aquela \textit{tag}.

\subsection{Através de Atributos}

Utilizando a barra de busca que se encontra na página que lista todos os documentos da aplicação, é possível realizar uma busca textual que recupera documentos que apresentem o texto informado nos seguintes elementos: Título, Descrição, Implementação ou Links Relacionados.

Além disso, é possível visualizar todos os documentos criados por um usuário ao se clicar no avatar do mesmo em qualquer tela do sistema.

https://www.dropbox.com/s/umzjanf5442zeai/Screenshot\%202015-11-11\%2015.03.38.png?dl=0

ou

https://www.dropbox.com/s/9qltf1b0chevyu8/Screenshot\%202015-11-11\%2015.03.48.png?dl=0

\section{Login com GitHub}

Para que desenvolvedores sejam identificados como autores dos documentos que criarem, é necessário que se autentiquem na aplicação com sua conta do GitHub. Isso também os permite importar código de repositórios privados para dentro da aplicação, tendo em vista que o GitHub exige autorização do próprio usuário para fornecer código privado.

Além disso, usuários logados podem utilizar a funcionalidade de ``Curtir'' documentos, descrita na próxima seção.

\section{``Curtir'' Documentos}

Usuários autenticados podem utilizar esta funcionalidade, que permite adicionar uma ``curtida'' a um documento. Ao fazer isso, todos os outros usuários da aplicação podem ver que aquele desenvolvedor ``curtiu'' aquele documento tanto na página do documento em si quando na página de listagem de documentos.

https://www.dropbox.com/s/zhvxgg3ryh1x96x/Screenshot\%202015-11-11\%2015.32.12.png?dl=0

https://www.dropbox.com/s/xlrj1745vd3baqm/Screenshot\%202015-11-11\%2015.32.54.png?dl=0

Esta funcionalidade permite ao usuário informar que o documento curtido foi útil para ele de alguma forma ou que pode ser útil para outros usuários, estimulando sua consulta por estes.

\section{Limitações da Ferramenta}

A aplicação ainda possui algumas limitações em suas funcionalidades. A respeito da importação, a ferramenta não suporta no momento a importação via \textit{links} de código que se encontram em outros ramos (\textit{branches}) de desenvolvimento do repositório.

Além disso, a ferramenta não trata mudanças que possam ter ocorrido na fonte do código importado. Ou seja, uma vez importado, mudanças ocorridas no código disponível no GitHub não se refletem na aplicação.

% IM: falar das vantagens e desvantagens? /\
