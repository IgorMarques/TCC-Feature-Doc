\chapter{Projeto de Solução}

Com base nos requisitos então elicitados, pôde-se ter uma noção clara de como seria feita a implementação da aplicação e este processo teve início. Ao longo dele, feedback a respeito da estética da ferramenta foi constantemente colhido com a equipe de desenvolvedores da empresa júnior bem como de parceiros do grupo de pesquisa do qual o autor faz parte.

\section{Listagem de Documentos}

Na tela inicial da aplicação é possível ver uma lista com todos os documentos já criados. Cada documento é exibido com seu título, descrição, lista de \textit{tags} e autor (caso tenha se identificado durante a criação do documento).

https://www.dropbox.com/s/4rput1kdwxxiknj/Screenshot%202015-11-05%2016.08.17.png?dl=0

Ao fim da lista de documentos, é possível ver quantos documentos se encontram nela e também uma lista das \textit{tags} presentes, incluindo sua quantidade de ocorrências dentro do grupo de documentos exibidos.

https://www.dropbox.com/s/bdhexdhgwrgo6fl/Screenshot%202015-11-05%2016.08.28.png?dl=0

ou

https://www.dropbox.com/s/us93gna8m8k4hkb/Screenshot%202015-11-05%2016.08.48.png?dl=0

ou

https://www.dropbox.com/s/3wm1tq2s6gqsrnq/Screenshot%202015-11-05%2016.08.59.png?dl=0

\section{Criação de um Documento}

Qualquer usuário da aplicação pode criar um documento.
% IM: tá ruim esse parágrafo, eu sei.

https://www.dropbox.com/s/nii8vbjzphgce0r/Screenshot%202015-11-11%2013.28.10.png?dl=0

\subsection{Elementos de um documento}

Um documento é composto por 5 elementos, todos informados pelo seu autor:

\begin{itemize}
  \item Título: Título do documento. Deve fornecer uma noção sobre o que aquele documento descreve.
  \item Descrição: Descrição curta. Um resumo sobre o que o documento descreve.
  \item Implementação: Procedimento sobre como implementar ou executar a funcionalidade descrita. Possui suporte a linguagem de marcação e importação de código armazenado no GitHub.
  \item Links Relacionados: Lista de links relevantes para o documento, com descrição sobre cada.
  \item Tags: Lista de tags sobre o assunto do documento
\end{itemize}

Sobre o elemento Implementação, a próxima subseção o descreve em detalhes.

\subsection{Implementação}

Este é o elemento mais importante de um documento e é utilizado para que o usuário descreva um passo-a-passo sobre
como realizar a implementação de uma funciondalidade, ou deixe registrado determinado procedimento. Dentre suas funcionalidades que permitem um uso aprimorado por parte de equipes de desenvolvimento de software está a importação de código hospedago no GitHub.

Esta importação quando se cola algum link do GitHub dentro do editor de texto deste campo. A seguir está uma descrição sobre os tipos de importações possíveis e seus respectivos resultados.

% IM: To do
% IM: To do
% IM: To do
% IM: To do
% IM: To do
\begin{itemize}
  \item Importação de Arquivo
  \item Importação de Linha Arquivo
  \item Importação de Intervalos de Linha Arquivo
  \item Importação de Commit
  \item Importação de Pull Request
\end{itemize}
% IM: To do
% IM: To do
% IM: To do
% IM: To do
% IM: To do

Todo código importado para aplicação já vem formatado na linguagem de marcação utilizada. Caso o arquivo importado seja de uma extensão conhecida, é adicionada a marcação para código da linguagem referente àquela extensão.

Sobre a linguagem de marcação utilizada, a próxima subseção a descreve, bem como suas vantagens para o contexto.

\subsection{Markdown}
\subsection{Exibição do Documento}

https://www.dropbox.com/s/dh6fao0hidf6dxw/Screenshot%202015-11-11%2013.33.45.png?dl=0

https://www.dropbox.com/s/m4x930cld84ga13/Screenshot%202015-11-11%2013.34.03.png?dl=0

\section{Recuperação de Documentos}

\subsection{Através de Tags}
\subsection{Através de Atributos}
% falar do "your twydies"

\section{Login com GitHub}

Para que os usuários possam importar informações de repositórios privados, é necessário que eles façam login [...]
% falar tbm da identificacao
% falar do your twydies

\section{"Curtir" Documentos}

to do
% \subsubsection

Limitações da Ferramenta

% nao importa código de branches
% nao liga para mudança nos arquivos
