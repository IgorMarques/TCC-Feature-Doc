\chapter{Introdução}

Desenvolvedores de software precisam reutilizar o conhecimento aprendido por outros desenvolvedores, de modo a proporcionar um melhor desempenho para toda organização~\cite{Druker1993}~\cite{Wiig2003}.
É esperado que os desenvolvedores melhorem continuamente seu trabalho em um processo em culmina na melhora significativa da sua empresa~\cite{Kavitha2011}.

Em equipes de desenvolvimento de software, o reuso de conhecimento ocorre utilizando uma variedade de artefatos, como código fonte, requisitos, modelos, dados e padrões~\cite{Levy2009}, bem como através de interações face a face, comunicação escrita e também via o repasse de referências de documentação, links, dentre outros~\cite{Storey2014}~\cite{Olson2000}~\cite{CubraniC2004}.

Neste quesito, a gerência de conhecimento, encarregada da elicitação, armazenamento, gerenciamento e reuso do conhecimento~\cite{Levy2009}.
Dentre as formas típicas de reuso, se encontra, por exemplo, como se deu a implementação de uma determinada funcionalidade em um determinado projeto de software.

A troca de informações é um mecanismo fundamental para que o reuso de conhecimento ocorra de maneira eficiente em equipes de desenvolvimento. Desenvolvedores, principalmente em fase de aprendizado de uma determinada tecnologia, comumente passam por situações e problemas semelhantes. Quando se deparam com algum tipo de adversidade, procuram por alguma fonte de informação capaz de auxiliá-los por tal problema. Existe então o desperdício de tempo de se recuperar tal informação (muitas vezes, interferindo nas atividades de um colega de trabalho) para se resolver um problema que já é de conhecimento da equipe dado alguma experiência anterior.

Com a falta de uma boa gerência de conhecimento, é comum desenvolvedores implementarem funcionalidades semelhantes em diferentes contextos usando abordagens \textit{ad-hoc}~\cite{SangMok2011}. Essas novas implementações, principalmente se feitas por desenvolvedores menos experientes, tendem a ser menos otimizadas e elegantes que soluções já previamente implementadas pela equipe, podendo gerar um débito técnico\footnote{Metáfora para as eventuais consequências de uma implementação não apropriada de determinada tarefa e que pode causar problemas para a manutenção do software no futuro.} a longo prazo.

Desenvolvedores mais experientes em um determinado projeto tendem a atuar como mentores~\cite{CubraniC2004} e tal ato, como atividade de gerência de conhecimento, tende a consumir recursos~\cite{Wiig2003}. Um dos pilares da gerência de conhecimento é a melhoria da produtividade através do compartilhamento e transferência eficiente de conhecimento, que tende a consumir tempo e às vezes sendo até impossível de se realizar~\cite{Levy2009}.

Muitas das interações entre desenvolvedores são informais e não há registro que possa ser consultado para propocionar o reuso do conhecimento trocado~\cite{Olson2000}. Um agravante tipicamente é a frequente rotatividade de membros em equipes. Nesses casos, a saída de um membro da equipe que detém determinado conhecimento pode prejudicar toda organização. De fato, a falta de uma abordagem mais sistemática para proporcionar o reuso de conhecimento pode estar associada ao fracasso de projetos em organizações~\cite{Hall2008}.

Assim, a elaboração de uma ferramenta capaz de agregar referências de código a soluções pode trazer enormes benefícios a equipes de desenvolvimento~\cite{CubraniC2004}. Este estudo propôs a elaboração de tal ferramenta. Esta atua como um catálogo, agregando referências de código, informações externas e comentários fornecidos por desenvolvedores da equipe com o intuito de auxiliar outros desenvolvedores a buscar em fontes da própria equipe (ou de confiança da mesma) como se deram implementações de funcionalidades já feitas em outros projetos.

\section{Exemplo de Uso da Ferramenta}

Um determinado desenvolvedor João, experiente dentro de sua equipe, percebe que muitos membros novatos procuram sua ajuda para implementar uma funcionalidade de exportação de uma página web para um documento em formato PDF. João então pode fazer uso da ferramenta, para registrar como se realiza a implementação de tal funcionalidade solicitada. Ele informa um título, descrição curta (para facilitar futuras buscas) e uma descrição de como se dá tal implementação. Como exemplificação em formato de código, o desenvolvedor pode utilizar de links de trechos de código disponívels artefatos de código já implementados pela sua equipe. Esses links são renderizados no editor de texto, sem a necessidade de João copiar e colar o código em si dentro do editor. João também pode complementar sua descrição adicionando links para outras páginas web (respostas de sites como Stack Overflow, outros tutoriais, etc...) em forma de anexo da documentação. Por fim, João informa tags relevantes ao artefato que está produzindo, facilitando a sua recuperação por parte de outros membros de sua equipe.

Outros desenvolvedores, a partir de então, podem recorrer diretamente a ferramenta quando precisarem implementar a funcionalidade de exportação de página em formato PDF. Dessa forma, João poderá se dedicar mais a outras atividades de seu dia-a-dia de trabalho.

\section{Objetivos e Perguntas de Pesquisa}

Este estudo descreve a elaboração de tal ferramenta e sua avaliação em dois contextos. O primeiro é o de uma equipe real de desenvolvimento de software composta por alunos da Universidade Federal do Rio Grande do Norte e o segundo contexto é o de alunos de graduação da área de computação da mesma universidade enquanto cursavam a disciplina de Desenvolvimento Colaborativo de Software no segundo semestre de 2015. A ferramenta foi utilizada para se auxiliarem durante o desenvolvimento do projeto final de disciplina.

O objetivo geral deste trabalho é responder as seguintes perguntas:

\begin{enumerate}
  \item A ferramenta proposta oferece suporte à maneira como conhecimento é trocado em equipes de software?
  \item Que outras funcionalidades podem ser agregadas a ferramenta para melhorar este suporte, caso exista?
  \item Quais são os casos potenciais de uso da ferramenta segundo os desenvolvedores?
\end{enumerate}

Para responder as perguntas de pesquisa propostas, foi feito aplicado um questionário com usuários de ferramenta.

Concluiu-se que, para a pergunta de pesquisa 1, sim, a ferramenta oferece suporte à troca de conhecimento entre desenvolvedores. Para a pergunta de pesquisa 2, um melhor suporte a pré-visualização e formatação e dos documentos podem aprimorar o suporte existente. Para a pergunta de pesquisa 3, os participantes informaram que sendo adivulgação de tutoriais ou soluções para suas equipes e registro pessoal sobre como o próprio desenvolvedor solucionou determinado problema seriam casos de uso nos quais utilizariam a ferramenta em sua rotina.

\section{Estrutura do Trabalho}

O Capítulo 2 apresenta uma revisão da literatura relacionada a este trabalho, focando no uso de mídias pelo engenheiro de software, os efeitos da rotatividade em equipes de software, além de gestão e recuperação de informação. O Capítulo 3 explica em detalhes os procedimentos tomados para a construção e análise da ferramenta tendo como base os conhecimentos citados no capítulo anterior. O Capítulo 4 descreve em detalhes a ferramenta elaborada. O Capítulo 5 realiza a análise dos resultados dos experimentos bem como discute seus resultados. O Capítulo 6 aborda as considerações finais sobre o trabalho.

